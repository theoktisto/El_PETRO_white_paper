%
\section{Condiciones técnicas}
\label{sec:annexa}
%%
%Anexo A: Condiciones técnicas
%
El PETRO, aunque podría terminar siendo una moneda, podría empezar siendo un criptoactivo diferente, una especie de ``criptopagaré'' denominado en petróleo. En fases posteriores, el PETRO podría evolucionar a divisa fuerte, en la medida en que sea aceptado como medio de intercambio y en la medida en que otras materias primas se sumen a su respaldo.

El PETRO, al ser un criptoactivo con respaldo, es muy diferente de las criptomonedas más populares, como el bitcoin, el ethereum o el litecoin, cuyos valores son fiduciarios. El valor mínimo del PETRO será el valor de canje por los bienes de respaldo. 

En su faceta de criptoactivo, el PETRO no debe ser minado por entes privados sino pre-minado y emitido de acuerdo con la política monetaria que el Estado Venezolano decida. Debe ir sobre una estructura de datos criptográfica (un blockchain) para generar confianza, pero permitir la minería digital de una moneda con respaldo de materias primas podría significar regalar dichas materias primas. 

La conversión en un mercado secundario (PETRO Exchange) del PETRO a otras divisas, así como el cambio de dueño de cada PETRO, debe hacerse de modo registrado, transparente y público. Se debe implementar sin las características de anonimidad que caracterizan a las criptomonedas más populares.

