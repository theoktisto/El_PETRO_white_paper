%
\section{Riesgos}
\label{sec:riesgo}
%
Se identifican estos riesgos principales.

    1) Implementación poco transparente de la blockchain, de manera que aparezcan o minen PETROS fuera de la emisión original (FRAUDE DE ORIGEN).
    2) Compra internacional de PETROS por parte de agentes económicos indeseables.
    3) PETRO Exchange permitirá poner condiciones de membresía, para evitar la participación de fondos buitre y otros inversionistas que busquen maniobrar en perjuicio de la economía venezolana. El PETRO Exchange funcionará sobre la plataforma de blockchain, donde quedarán registradas todas las transacciones de forma transparente.
    4) Fuga de capitales, si las divisas obtenidas de la venta de PETROS por parte de sus tenedores venezolanos no se usan para importaciones.

Con esta propuesta se elimina la percepción de que el bolívar se devaluará indefinidamente, que será sustituida por la percepción de que fluctuará frente a otras divisas según fluctúe el precio del petróleo. Esto eliminará presión para refugiar los capitales en otras divisas.

Se propone crear adicionalmente un mecanismo/legislación de control de capitales. A través del Blockchain se podrá saber quiénes compran PETROS en bolívares y quienes los venden en el exterior y a quién. Es necesario evitar la fuga de grandes capitales o la fuga masiva de pequeños capitales, por ejemplo, exigiendo a los vendedores venezolanos de PETROS algún tipo de comprobante del uso de divisas para importaciones. El registro del blockchain permitirá una total transparencia y podrá ser usado como prueba en un tribunal.