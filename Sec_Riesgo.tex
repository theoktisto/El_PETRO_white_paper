%
\section{Riesgos}
\label{sec:riesgo}
%
Se identifican estos riesgos principales.
\begin{itemize}

    \item{1)} Implementación poco transparente de la blockchain, de manera que aparezcan o minen P-E-T-R-OS fuera de la emisión original (FRAUDE DE ORIGEN).
    \item{2)} Compra internacional de P-E-T-R-OS por parte de agentes económicos indeseables.
    \item{3)} P-E-T-R-O Exchange permitirá poner condiciones de membresía, para evitar la participación de fondos buitre y otros inversionistas que busquen maniobrar en perjuicio de la economía venezolana. El P-E-T-R-O Exchange funcionará sobre la plataforma de blockchain, donde quedarán registradas todas las transacciones de forma transparente.
    \item{4)} Fuga de capitales, si las divisas obtenidas de la venta de P-E-T-R-OS por parte de sus tenedores venezolanos no se usan para importaciones.
\end{itemize}

Con esta propuesta se elimina la percepción de que el bolívar se devaluará indefinidamente, que será sustituida por la percepción de que fluctuará frente a otras divisas según fluctúe el precio del petróleo. Esto eliminará presión para refugiar los capitales en otras divisas.

Se propone crear adicionalmente un mecanismo/legislación de control de capitales. A través del Blockchain se podrá saber quiénes compran P-E-T-R-OS en bolívares y quienes los venden en el exterior y a quién. Es necesario evitar la fuga de grandes capitales o la fuga masiva de pequeños capitales, por ejemplo, exigiendo a los vendedores venezolanos de P-E-T-R-OS algún tipo de comprobante del uso de divisas para importaciones. El registro del blockchain permitirá una total transparencia y podrá ser usado como prueba en un tribunal.