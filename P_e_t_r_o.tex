\documentclass[spanish,letterpaper,11pt]{article}
%\usepackage{tweaklist}
\usepackage{url}
\usepackage{xspace}
\usepackage{graphicx}
\usepackage{multicol}
\usepackage{subfig}
\usepackage{amsmath}
\usepackage{amssymb}
\usepackage[width=170mm,top=18mm,bottom=22mm,includeheadfoot]{geometry}
\usepackage{booktabs}
\usepackage{array}
\usepackage{verbatim}
\usepackage{caption}
\usepackage{natbib}
\usepackage{float}
\usepackage[utf8]{inputenc}
\usepackage[english,main=spanish]{babel}
\usepackage[T1]{fontenc}
\usepackage{pdflscape}
\usepackage{mathtools}
\usepackage[usenames,dvipsnames]{xcolor}
\usepackage{afterpage}
\usepackage[toc,page]{appendix}
\definecolor{lightyellow}{rgb}{1,0.98,0.9}

\DeclarePairedDelimiter{\ceil}{\lceil}{\rceil}
\newcommand*\eg{e.g.\@\xspace}
\newcommand*\Eg{e.g.\@\xspace}
\newcommand*\ie{i.e.\@\xspace}
%\renewcommand{\itemhook}{\setlength{\topsep}{0pt}  \setlength{\itemsep}{0pt}\setlength{\leftmargin}{15pt}}

\title{\bf El  P-E-T-R-O: Un criptoactivo seguro respaldado por el Estado venezolano con barriles de petróleo de las reservas}
\author{
     Dr. Unk N. Own (Life University)\\
}
\date{}
\begin{document}
\selectlanguage{spanish}

%\pagecolor{lightyellow}
\maketitle

\begin{abstract}
Propuesta para la estabilización de la economía venezolana en un plazo estimado de seis meses a un año, por la vía de la creación del criptoactivo llamado “P-E-T-R-O” y su poder para desplazar al dólar paralelo, siempre que su mercado primario sea en bolívares.
\end{abstract}

\setlength{\columnsep}{20pt}
%\begin{multicols}{1}

\tableofcontents

% ******************************** Main Matter *********************************

%\selectlanguage{spanish}
%
\section{Introducción}
\label{sec:intro}
%
El P-E-T-R-O tiene el poder para detener la perniciosa influencia del dólar paralelo, marcador de precios que cuenta con la complicidad de grandes grupos de importadores y comerciantes. Por otro lado, la Revolución tiene que evolucionar hacia un sistema monetario que sea la base de la protección del salario, lo que exige que se controle la devaluación y la especulación. Un activo con respaldo del petróleo y de otras materias primas, dadas las enormes reservas minerales que tiene Venezuela, puede perfectamente jugar ese papel.

\subsection{Criptomonedas y Criptoactivos}
\label{sec:cripto}

   
\subsection{Experiencias previas}
\label{sec:previous}

Smart contracts. Bitcoin, Ripple, Iota.


%
\section{Propuesta}
\label{sec:proposal}
%
\subsection{Condiciones económicas}
\label{sec:overview}
%
Las condiciones económicas básicas para que el PETRO detenga a corto plazo la influencia del dólar paralelo marcador de precios son las siguientes:
\begin{itemize}
\item{1)} Oferta de PETROS para su compra en Bolívares, sin escasez inducida. En otras palabras, el mercado primario del PETRO debe ser en Bolívares, y la oferta de PETROS en Bolívares no debe estar limitada por una cantidad que se quede corta sino por la demanda, es decir, se emiten todos los PETROS que demande la economía venezolana. Como se demuestra más adelante, esto es factible. Nunca se podrán comprar más PETROS que lo que permite el circulante (M1). Todo comprador externo de PETROS deberá cambiar a Bolívares previamente, al cambio indicado por el Estado Venezolano, no usando tasas de cambio ilegales y poco confiables.
%
\item{2)} Paridad monetaria {\bf{\it fija}} Bolívar-PETRO. El valor monetario del Bolívar se debe {\bf{\it anclar}} rígidamente al valor del PETRO (paridad monetaria fija), dejando que ambos fluctúen juntos frente a otras divisas según sea el valor del barril de petróleo en el mercado internacional. Dicha fluctuación será inevitable, pero no podrá ser controlada por un pequeño grupo a través de una página web. No se puede permitir una fluctuación libre entre PETROS y Bolívares porque el Bolívar corre el riesgo de hundirse frente al PETRO (Ley de Gresham). Al cambiar Bolívares por PETROS, los Bolívares deben salir del circulante, de lo contrarió, se estará dejando en el sistema esa cantidad en dinero inorgánico (inflación).
%   
\item{3)} Creación de un PETRO Exchange único. Es una institución donde se realizarán las transacciones entre tenedores venezolanos de PETROS y compradores internacionales de PETROS, presumiblemente clientes de PDVSA en su mayoría, aunque podría haber otros tipos de inversionistas. Así, los clientes de PDVSA podrían usar el PETRO para pagar la factura petrolera, junto a otras divisas que acepte PDVSA.
\end{itemize}

%
\section{Riesgos}
\label{sec:riesgo}
%
Se identifican estos riesgos principales.
\begin{itemize}

    \item{1)} Implementación poco transparente de la blockchain, de manera que aparezcan o minen P-E-T-R-OS fuera de la emisión original (FRAUDE DE ORIGEN).
    \item{2)} Compra internacional de P-E-T-R-OS por parte de agentes económicos indeseables.
    \item{3)} P-E-T-R-O Exchange permitirá poner condiciones de membresía, para evitar la participación de fondos buitre y otros inversionistas que busquen maniobrar en perjuicio de la economía venezolana. El P-E-T-R-O Exchange funcionará sobre la plataforma de blockchain, donde quedarán registradas todas las transacciones de forma transparente.
    \item{4)} Fuga de capitales, si las divisas obtenidas de la venta de P-E-T-R-OS por parte de sus tenedores venezolanos no se usan para importaciones.
\end{itemize}

Con esta propuesta se elimina la percepción de que el bolívar se devaluará indefinidamente, que será sustituida por la percepción de que fluctuará frente a otras divisas según fluctúe el precio del petróleo. Esto eliminará presión para refugiar los capitales en otras divisas.

Se propone crear adicionalmente un mecanismo/legislación de control de capitales. A través del Blockchain se podrá saber quiénes compran P-E-T-R-OS en bolívares y quienes los venden en el exterior y a quién. Es necesario evitar la fuga de grandes capitales o la fuga masiva de pequeños capitales, por ejemplo, exigiendo a los vendedores venezolanos de P-E-T-R-OS algún tipo de comprobante del uso de divisas para importaciones. El registro del blockchain permitirá una total transparencia y podrá ser usado como prueba en un tribunal.
%
\section{Factibilidad económica}
\label{sec:fact}
%
La emisión sin escasez inducida es perfectamente factible porque, a una paridad PETRO-bolívar razonable, digamos Bs. 100.000,00 por PETRO, la relación entre el circulante y las reservas petroleras monetizadas es de alrededor de 1:1000 a favor de las reservas. La compra de PETROS nunca va a llegar a cubrir el circulante (M1), quizás un 30\% de éste. Una emisión inicial (ICO) de 150 millones de PETROS es suficiente.

Venezuela produce mucho más de 150 millones de barriles cada año (El Presidente Maduro propuso 5.000 millones del Bloque 1 del Campo Ayacucho de la Faja, 30 veces esta cantidad). Sucesivas emisiones serán para reponer los PETROS que se van convirtiendo en divisas en el mercado secundario. Dichas emisiones deben guardar relación con el ritmo de producción petrolera, es decir, las condiciones del despacho al redimirse los PETROS por barriles de Petróleo.
%
\subsection{Ejercicio práctico 1}
%
Dado que un PETRO vale un barril de petróleo [GO 6346], a una paridad fija PETRO-Bolívar de Bs. 100.000,00 por PETRO y estando el petróleo venezolano a \$50 en el mercado internacional, la paridad indirecta del Bolívar con respecto el dólar será de Bs. 2.000,00 por dólar. Esto hace que el sueldo mínimo quede en alrededor de \$150. Esa misma paridad de Bs. 100.000,00 por PETRO, si el petróleo sube a \$100, determina una paridad indirecta Bolívar-dólar de Bs 1.000,00 por dólar y un sueldo mínimo de unos \$300. Es decir, debido a la paridad monetaria fija entre el PETRO y el Bolívar, estos fluctuaran juntos frente a otras divisas, pero no entre ambos. Esto permitiría rescatar el poder de compra con el salario mínimo nacional y el poder de compra asociado.

%
\section{Conclusiones}
\label{sec:conclu}
%
El PETRO puede estabilizar la economía venezolana y hacer insustancial al dólar paralelo de guerra económica si su mercado primario es en Bolívares. La venta directa de PETROS en divisa extranjera no tiene que ser mala \emph{per se}, pero es una simple venta a futuro de petróleo y otras materias primas, que sólo serían útiles para ayudar al Estado a pagar deudas, pero no incide contundentemente en el bienestar de los venezolanos. En cambio, su venta en bolívares a una paridad fija permite construir un gran bastión en defensa del salario de todos los venezolanos.

% *********************** Adding TOC and List of Figures ***********************

\bibliography{Biblio}
\bibliographystyle{plainnat}

%\end{multicols}
\renewcommand{\appendixname}{Anexos}
\begin{appendices} % Using appendices environment for more functunality

%
\section{Condiciones técnicas}
\label{sec:annexa}
%%
%Anexo A: Condiciones técnicas
%
El P-E-T-R-O, aunque podría terminar siendo una moneda, podría empezar siendo un criptoactivo diferente, una especie de "criptopagaré" denominado en petróleo. En fases posteriores, el P-E-T-R-O podría evolucionar a divisa fuerte, en la medida en que sea aceptado como medio de intercambio y en la medida en que otras materias primas se sumen a su respaldo.

El P-E-T-R-O, al ser un criptoactivo con respaldo, es muy diferente de las criptomonedas más populares, como el bitcoin, el ethereum o el litecoin, cuyos valores son fiduciarios. El valor mínimo del P-E-T-R-O será el valor de canje por los bienes de respaldo. 

En su faceta de criptoactivo, el P-E-T-R-O no debe ser minado por entes privados sino pre-minado y emitido de acuerdo con la política monetaria que el Estado decida. Debe ir sobre una estructura de datos criptográfica (un blockchain) para generar confianza, pero permitir la minería digital de una moneda con respaldo de materias primas podría significar regalar dichas materias primas. 

La conversión en un mercado secundario (P-E-T-R-O Exchange) del P-E-T-R-O a otras divisas, así como el cambio de dueño de cada P-E-T-R-O, debe hacerse de modo registrado, transparente y público. Se debe implementar sin las características de anonimidad que caracterizan a las criptomonedas más populares.


%
\section{Proceso de revaluación}
\label{sec:annexb}
%%
%Anexo B: Proceso de revaluación
%
Gracias al eventual respaldo del bolívar con el P-E-T-R-O, puede revertirse el proceso de devaluación y, más bien al contrario, iniciarse un proceso de revaluación. Un proceso de revaluación recoge parcialmente el agua derramada.

Un proceso de revaluación es estímulo a que los comerciantes saquen sus productos a la venta en vez de esconderlos, porque mientras más esperan, más barato tienen que vender. Es estímulo a que los ciudadanos no quieran gastar su dinero tan pronto lo obtienen, salvo en lo imprescindible, porque si esperan, comprarán más barato. Habrá más compradores de bolívares, con divisas, ante la expectativa de ganar. Eso ayuda en el proceso de revaluación. Va frenando la salida de productos al exterior. Con un bolívar fuerte, vislumbrado sabiamente por el Comandante Hugo Chávez, más bien retornará el flujo de productos hacia Venezuela. 

No podemos desviarnos del objetivo de ser cada vez más productivos, pero no debemos cifrar esperanzas en la estrategia monetaria devaluacionista como estimulador de exportaciones, especialmente en el contexto de una guerra económica. Casi lo único que no podrán bloquear desde el exterior serán las ventas de petróleo.

%\include{Appendix2/appendix2}

\end{appendices}

\end{document}
