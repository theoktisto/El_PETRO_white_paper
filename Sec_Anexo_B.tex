%
\section{Proceso de revaluación}
\label{sec:annexb}
%%
%Anexo B: Proceso de revaluación
%
Gracias al eventual respaldo del bolívar con el PETRO, puede revertirse el proceso de devaluación y, más bien al contrario, iniciarse un proceso de revaluación. Un proceso de revaluación recoge parcialmente el agua derramada.

Un proceso de revaluación es estímulo a que los comerciantes saquen sus productos a la venta en vez de esconderlos, porque mientras más esperan, más barato tienen que vender. Es estímulo a que los ciudadanos no quieran gastar su dinero tan pronto lo obtienen, salvo en lo imprescindible, porque si esperan, comprarán más barato. Habrá más compradores de bolívares, con divisas, ante la expectativa de ganar. Eso ayuda en el proceso de revaluación. Va frenando la salida de productos al exterior. Con un bolívar fuerte, vislumbrado sabiamente por el Comandante Hugo Chávez, más bien retornará el flujo de productos hacia Venezuela. 

No podemos desviarnos del objetivo de ser cada vez más productivos, pero no debemos cifrar esperanzas en la estrategia monetaria devaluacionista como estimulador de exportaciones, especialmente en el contexto de una guerra económica. Casi lo único que no podrán bloquear desde el exterior serán las ventas de petróleo.