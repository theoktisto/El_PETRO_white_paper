%
\section{Propuesta}
\label{sec:proposal}
%
\subsection{Condiciones económicas}
\label{sec:overview}
%
Las condiciones económicas básicas para que el PETRO detenga a corto plazo la influencia del dólar paralelo marcador de precios son las siguientes:
\begin{itemize}
\item{1)} Oferta de PETROS para su compra en Bolívares, sin escasez inducida. En otras palabras, el mercado primario del PETRO debe ser en Bolívares, y la oferta de PETROS en Bolívares no debe estar limitada por una cantidad que se quede corta sino por la demanda, es decir, se emiten todos los PETROS que demande la economía venezolana. Como se demuestra más adelante, esto es factible. Nunca se podrán comprar más PETROS que lo que permite el circulante (M1). Todo comprador externo de PETROS deberá cambiar a Bolívares previamente, al cambio indicado por el Estado Venezolano, no usando tasas de cambio ilegales y poco confiables.
%
\item{2)} Paridad monetaria {\bf{\it fija}} Bolívar-PETRO. El valor monetario del Bolívar se debe {\bf{\it anclar}} rígidamente al valor del PETRO (paridad monetaria fija), dejando que ambos fluctúen juntos frente a otras divisas según sea el valor del barril de petróleo en el mercado internacional. Dicha fluctuación será inevitable, pero no podrá ser controlada por un pequeño grupo a través de una página web. No se puede permitir una fluctuación libre entre PETROS y Bolívares porque el Bolívar corre el riesgo de hundirse frente al PETRO (Ley de Gresham). Al cambiar Bolívares por PETROS, los Bolívares deben salir del circulante, de lo contrarió, se estará dejando en el sistema esa cantidad en dinero inorgánico (inflación).
%   
\item{3)} Creación de un PETRO Exchange único. Es una institución donde se realizarán las transacciones entre tenedores venezolanos de PETROS y compradores internacionales de PETROS, presumiblemente clientes de PDVSA en su mayoría, aunque podría haber otros tipos de inversionistas. Así, los clientes de PDVSA podrían usar el PETRO para pagar la factura petrolera, junto a otras divisas que acepte PDVSA.
\end{itemize}
