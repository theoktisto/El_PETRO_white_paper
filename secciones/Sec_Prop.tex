%
\section{Propuesta}
\label{sec:proposal}
%
\subsection{Condiciones económicas}
\label{sec:overview}
%
Las condiciones económicas básicas para que el P-E-T-R-O detenga a corto plazo la influencia del dólar paralelo marcador de precios son las siguientes:
\begin{itemize}
\item{1)} Oferta de P-E-T-R-OS para su compra en bolívares, sin escasez inducida. En otras palabras, el mercado primario del P-E-T-R-O debe ser en bolívares, y la oferta de P-E-T-R-OS en bolívares no debe estar limitada por una cantidad que se quede corta sino por la demanda, es decir, se emiten todos los P-E-T-R-OS que demande la economía venezolana. Como se demuestra más adelante,  esto es factible, Nunca se podrán comprar más P-E-T-R-OS que lo que permite el circulante (M1). Todo comprador externo de P-E-T-R-OS deberá cambiar a Bolívares previamente, al cambio indicado por el Estado, no usando tasas de cambio ilegales y poco confiables.
%
\item{2)} Paridad FIJA bolívar-P-E-T-R-O. El valor del bolívar se debe atar al valor del P-E-T-R-O (paridad constante), dejando que ambos fluctúen juntos frente a otras divisas según sea el valor del barril de petróleo en el mercado internacional. Dicha fluctuación será inevitable, pero no podrá ser controlada por un pequeño grupo a través de una página web. No se puede permitir una fluctuación libre entre P-E-T-R-OS y bolívares porque el bolívar corre el riesgo de hundirse frente al P-E-T-R-O (Ley de Gresham). Al cambiar bolívares por P-E-T-R-OS, los bolívares deben salir del circulante.
%   
\item{3)} Creación de un P-E-T-R-O Exchange único. Es una institución donde se realizarán las transacciones entre tenedores venezolanos de P-E-T-R-OS y compradores internacionales de P-E-T-R-OS, presumiblemente clientes de PDVSA en su mayoría, aunque podría haber otros tipos de inversionistas. Así, los clientes de PDVSA podrían usar el P-E-T-R-O para pagar la factura P-E-T-R-Olera, junto a otras divisas que acepte PDVSA.
\end{itemize}
