%
\section{Factibilidad económica}
\label{sec:fact}
%
La emisión sin escasez inducida es perfectamente factible porque, a una paridad PETRO-bolívar razonable, digamos Bs. 100.000,00 por PETRO, la relación entre el circulante y las reservas petroleras monetizadas es de alrededor de 1:1000 a favor de las reservas. La compra de PETROS nunca va a llegar a cubrir el circulante (M1), quizás un 30\% de éste. Una emisión inicial (ICO) de 150 millones de PETROS es suficiente.

Venezuela produce mucho más de 150 millones de barriles cada año (El Presidente Maduro propuso 5.000 millones del Bloque 1 del Campo Ayacucho de la Faja, 30 veces esta cantidad). Sucesivas emisiones serán para reponer los PETROS que se van convirtiendo en divisas en el mercado secundario. Dichas emisiones deben guardar relación con el ritmo de producción petrolera, es decir, las condiciones del despacho al redimirse los PETROS por barriles de Petróleo.
%
\subsection{Ejercicio práctico 1}
%
Dado que un PETRO vale un barril de petróleo [GO 6346], a una paridad fija PETRO-Bolívar de Bs. 100.000,00 por PETRO y estando el petróleo venezolano a \$50 en el mercado internacional, la paridad indirecta del Bolívar con respecto el dólar será de Bs. 2.000,00 por dólar. Esto hace que el sueldo mínimo quede en alrededor de \$150. Esa misma paridad de Bs. 100.000,00 por PETRO, si el petróleo sube a \$100, determina una paridad indirecta Bolívar-dólar de Bs 1.000,00 por dólar y un sueldo mínimo de unos \$300. Es decir, debido a la paridad monetaria fija entre el PETRO y el Bolívar, estos fluctuaran juntos frente a otras divisas, pero no entre ambos. Esto permitiría rescatar el poder de compra con el salario mínimo nacional y el poder de compra asociado.
