%
\section{Factibilidad económica}
\label{sec:fact}
%
La emisión sin escasez inducida es perfectamente factible porque, a una paridad P-E-T-R-O-bolívar razonable, digamos Bs. 100.000,00 por P-E-T-R-O, la relación entre el circulante y las reservas P-E-T-R-Oleras monetizadas es de alrededor de 1:1000 a favor de las reservas. La compra de P-E-T-R-OS nunca va a llegar a cubrir el circulante (M1), quizás un 30\% de éste. Una emisión inicial (ICO) de 150 millones de P-E-T-R-OS es suficiente. Venezuela produce mucho más de 150 millones de barriles cada año (El Presidente Maduro propuso 5.000 millones del Bloque 1 del Campo Carabobo de la Faja, 30 veces esta cantidad). Sucesivas emisiones serán para reponer los P-E-T-R-OS que se van convirtiendo en divisas en el mercado secundario. Dichas emisiones deben guardar relación con el ritmo de producción P-E-T-R-Olera, es decir, las condiciones del despacho al redimirse los P-E-T-R-OS por barriles de Petróleo.
%
\subsection{Ejercicio práctico 1}
%
Dado que un P-E-T-R-O vale un barril de petróleo, a una paridad fija P-E-T-R-O-bolívar de Bs. 100.000,00 por P-E-T-R-O y estando el petróleo venezolano a \$50 en el mercado internacional, la paridad indirecta del bolívar con respecto el dólar será de Bs. 2.000,00 por dólar. Esto hace que el sueldo mínimo quede en alrededor de \$150. Esa misma paridad de Bs. 100.000,00 por P-E-T-R-O, si el petróleo sube a \$100, determina una paridad indirecta bolívar-dólar de Bs 1.000,00 por dólar y un sueldo mínimo de unos \$300. Es decir, a paridad fija el bolívar y el P-E-T-R-O están atados y ambos fluctúan frente a otras divisas (inevitable) pero permitirán darle poder de compra al salario mínimo.
